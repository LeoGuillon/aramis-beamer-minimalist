%%%%%%%%%%%%%%%%%%%%%%%%%%%%%%%%%%%%%%%%%
% Beamer Presentation - LaTeX Template
% Version 2.0 (March 8, 2022)
% Original Template: https://www.LaTeXTemplates.com
% License: CC BY-NC-SA 4.0
%%%%%%%%%%%%%%%%%%%%%%%%%%%%%%%%%%%%%%%%%

\documentclass[
	11pt, % Default font size
	aspectratio=169, % Set 16:9 aspect ratio
]{beamer}


% Packages
\usepackage[T1]{fontenc}
\usepackage{subcaption}
\usepackage{tikz}
\usepackage{lipsum} % useful for creating some dummy text
\usepackage[base]{babel} % comment if you intend to create a presentation in french
% \usepackage[french]{babel} % uncomment if you intend to create a presentation in french

% Layout theme
\usetheme[
  % showmaxslides, % displays the total page number in the page numbering
  % showsectiontoc, % displays a table of contents at the beginning of each section
  serif,  % uses a serif font for body text ;
          % depending on your TeX compiling engine, it will be different :
          %   on LuaTeX and XeTeX : EB Garamond
          %   on pdfTeX : Times New Roman
          % NB : if you use any amount of math in your presentation, it is advised to use a serif font
  nexa  % uses Nexa font for structure text instead of default Trebuchet MS ;
        % WARNING : you must build with LuaTeX or XeTeX, otherwise the default beamer font (e.g. sans-serif Computer Modern) will be used.
]{Aramis}

% Paths and misc stuff
\graphicspath{{img/}}

%------------------
%	PRESENTATION INFORMATION
%------------------

\title{Title : usually longer than you would expect for a research paper, so make sure it fits}
\subtitle{Subtitle : shorter than the title}
\author{Author\inst{1}, another author\inst{2}}
\institute{\inst{1}ARAMIS Lab, \inst{2}Another institute}
\date{\today}

%------------------

% \includeonlyframes{titlepage}

\begin{document}

%------------------
%	TITLE SLIDE
%------------------

\begin{frame}[plain, label=titlepage]
	\titlepage % Displays the title slide
\end{frame}

%------------------
%	TABLE OF CONTENTS SLIDE
%------------------

\begin{frame}
	\frametitle{Presentation Outline} % Slide title
	\tableofcontents[hideallsubsections] % Displays the table of contents
\end{frame}

%------------------
%	BODY SECTIONS
%------------------

\section{Text}

\begin{frame}
  \frametitle{Plain text}

  \lipsum[1][1-5]
  \vfill
  \textbf{This is some bolt text.}
  \vfill
  \textit{This is some italic text.}

\end{frame}

\begin{frame}
  \frametitle{Some maths}

  You can display some math blocks like this :
  \[\mathcal{N}(x~|~\mu, \sigma) = \frac{1}{\sqrt{2\pi\sigma^2}}\exp\left\{-\frac{(x-\mu)^2}{2\sigma^2}\right\},\]
  or also write inline math like this : $f(x) = \mathcal{N}(x~|~\mu, \sigma)$.

\end{frame}

\begin{frame}

  Here’s a plain frame, without any title.

\end{frame}

\section{Lists}

\subsection{Bullet lists}
\begin{frame}
  \frametitle{Bullet list}

  \begin{itemize}
    \item First point
    \item Second point
    \item Third point
  \end{itemize}

\end{frame}

\subsection{Numbered lists}
\begin{frame}{Numbered list}
  \begin{enumerate}
    \item First point
    \item Second point
    \item Third point
  \end{enumerate}

\end{frame}



\section{Figures and stuff}

\begin{frame}
  \frametitle{Figure}

  \begin{figure}
    \includegraphics[height=.6\paperheight]{logos/logo_ARAMISLAB.png}
    \caption{A sample figure}
  \end{figure}

\end{frame}

\begin{frame}
  \frametitle{Table}

  \begin{table}
    \begin{tabular}{|c|c|}
      \hline
      column 1 & column 2\\
      \hline
      some stuff here & and also there\\
      \hline
      another column & one last cell\\
      \hline
    \end{tabular}
    \caption{A sample table}
  \end{table}

\end{frame}


\section{Bibliography}

\begin{frame}
  \frametitle{References}

\end{frame}

\end{document}
