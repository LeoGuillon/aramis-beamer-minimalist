% —————————————————————————————————————————————————————————————————————————————————
% Aramis beamer template
% Version 1.0 (March 14, 2025)
% —————————————————————————————————————————————————————————————————————————————————

\documentclass[
  11pt, % Default font size
  aspectratio=169, % Set 16:9 aspect ratio
]{beamer}

% Packages
\usepackage[T1]{fontenc}
\usepackage{subcaption} % for subfigures
\usepackage{booktabs} % for better-looking tables
\usepackage{acronym} % for acronyms management

\usepackage{tikz} % mandatory for the titlepage to be correctly displayed
\usetikzlibrary{calc}

\usepackage{lipsum} % useful for creating some dummy text

\usepackage[base]{babel} % comment if you intend to create a presentation in french
% \usepackage[french]{babel} % uncomment if you intend to create a presentation in french

% Layout theme
\usetheme[
  % showmaxslides, % displays the total page number in the page numbering
  % showsectiontoc, % displays a table of contents at the beginning of each section
  serif,  % uses a serif font for body text ;
  % depending on your TeX compiling engine, it will be different :
  %   on LuaTeX and XeTeX : EB Garamond
  %   on pdfTeX : Times New Roman
  % NB : if you use any amount of math in your presentation, it is advised to use a serif font
  nexa  % uses Nexa font for structure text instead of default Trebuchet MS ;
  % WARNING : you must build with LuaTeX or XeTeX, otherwise the default beamer font (e.g. sans-serif Computer Modern) will be used.
]{Aramis}

% Paths and misc stuff
\graphicspath{{assets/img/}}

% —————————————————————————————————————————————————————————————————————————————————
% (PRESENTATION METADATA)
% —————————————————————————————————————————————————————————————————————————————————

\title{Title : usually longer than you would expect for a research paper, so make sure it fits}
\subtitle{Subtitle : shorter than the title}
\author{Author\inst{1}, another author\inst{2}}
\institute{\inst{1}ARAMIS Lab, \inst{2}Another institute}
\date{\today}

% —————————————————————————————————————————————————————————————————————————————————
% (SLIDES)
% —————————————————————————————————————————————————————————————————————————————————

\begin{document}

% —————————————————————————————————————————————————————————————————————————————————
% (ACRONYMS LOADING)

% NOTE: must be done within the document environment
% NOTE: we use the acrodef command so that there's no need to display
% an acronym table in our presentation

\acrodef{ICM}{Institut du cerveau}
\acrodef{MRI}{Magnetic resonance imaging}


% —————————————————————————————————————————————————————————————————————————————————
% (TITLE PAGE)

\begin{frame}[plain, label=titlepage]
  \titlepage
\end{frame}

% —————————————————————————————————————————————————————————————————————————————————
% (TABLE OF CONTENTS)

\begin{frame}{Presentation Outline}
  \tableofcontents[hideallsubsections]
\end{frame}

% —————————————————————————————————————————————————————————————————————————————————
% (BODY SLIDES)

% if you have a big presentation and would prefer a more modular structure,
% you can put your slides in src/slides and requestes them with \input

\section{Text}

\begin{frame}
  \frametitle{Plain text}

  Body text serif font : EB Garamond.
  \vfill
  \lipsum[1][1-5]
  \vfill
  \textbf{This is some bolt text.}
  \vfill
  \textit{This is some italic text.}
  \vfill
  \emph{This is some text with emphasis.}
  \vfill
  Typographic advice : never use \textbf{\textit{italic bold text}} or \underline{underline text} in presentation ; also, tend to use the \emph{emphasis} instead of \textit{italic} or \textbf{bold}.

\end{frame}

\begin{frame}{Code blocks}
  Monospace font : \texttt{Jetbrains Mono}.

  \vfill
  You can write inline code : \texttt{print("Hello world!")}.

  \vfill
  And also displays code blocks :

  % TODO: display a code block environment
\end{frame}

\begin{frame}
  \frametitle{Math blocks}

  Math font : $Garamond-math$.

  \vfill
  You can display some math blocks like this :
  \[\mathcal{N}(x~|~\mu, \sigma) = \frac{1}{\sqrt{2\pi\sigma^2}}\exp\left\{-\frac{(x-\mu)^2}{2\sigma^2}\right\},\]
  or also write inline math like this : $f(x) = \mathcal{N}(x~|~\mu, \sigma)$.

\end{frame}

\begin{frame}{Acronyms management}
  We can use defined acronyms : their first appareance will feature their definition, such as “Let's talk about \ac{MRI}."
  \vfill

  Then, their following appearances will automatically be as acronyms : "Let's talk again about \ac{MRI}."
\end{frame}

\begin{frame}

  Here’s a plain frame, without any title.

\end{frame}

\section{Lists}

\subsection{Bullet lists}
\begin{frame}
  \frametitle{Bullet list}

  \begin{itemize}
    \item First point
    \item Second point
    \item Third point
  \end{itemize}

\end{frame}

\subsection{Numbered lists}
\begin{frame}{Numbered list}
  \begin{enumerate}
    \item First point
    \item Second point
    \item Third point
  \end{enumerate}

\end{frame}

\section{Figures and stuff}

\begin{frame}
  \frametitle{Figure}

  \begin{figure}
    \includegraphics[height=.6\paperheight]{logos/logo_ARAMISLAB.png}
    \caption{A sample figure}
  \end{figure}

\end{frame}

\begin{frame}
  \frametitle{Table}

  \begin{table}
    \begin{tabular}{cc}
      \toprule
      column 1 & column 2\\
      \midrule
      some stuff here & and also there\\
      another column & one last cell\\
      \bottomrule
    \end{tabular}
    \caption{A sample table}
  \end{table}

\end{frame}

\section{Bibliography}

\begin{frame}
  \frametitle{References}

\end{frame}

% —————————————————————————————————————————————————————————————————————————————————
% (APPENDIX)

% the following slides will not be counted in the total frames number,
% if ever this option is enabled

\appendix

\end{document}
