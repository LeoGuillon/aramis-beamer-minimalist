%%%%%%%%%%%%%%%%%%%%%%%%%%%%%%%%%%%%%%%%%
% Beamer Presentation - LaTeX Template
% Version 2.0 (March 8, 2022)
% Original Template: https://www.LaTeXTemplates.com
% License: CC BY-NC-SA 4.0
%%%%%%%%%%%%%%%%%%%%%%%%%%%%%%%%%%%%%%%%%

%%%%%%%%%%%%%%%%%%%%%%%%%%%%%%%%%%%%%%%%%

\documentclass[
	11pt, % Default font size
	aspectratio=169, % Set 16:9 aspect ratio
]{beamer}


% Packages
\usepackage[T1]{fontenc}
\usepackage{subcaption}
\usepackage{tikz}
% \usepackage[french]{babel} % uncomment if you intend to create a presentation in french

% Layout theme ; Available options :
% - nexa : uses Nexa font instead of default Trebuchet MS
% - showmaxslides : displays the total page number in the page numbering
\usetheme[nexa]{Aramis}

% Paths and misc stuff
\graphicspath{{img/}}

%------------------
%	PRESENTATION INFORMATION
%------------------

\title{Title : usually longer than you would expect for a research paper, so make sure it fits}
\subtitle{Subtitle : shorter than the title}
\author{Surname Name} 
\institute{ARAMIS Lab} 
\date{\today} 

%------------------

\begin{document}

%------------------
%	TITLE SLIDE
%------------------

\begin{frame}
	\titlepage % Displays the title slide
\end{frame}

%------------------
%	TABLE OF CONTENTS SLIDE
%------------------

\begin{frame}
	\frametitle{Presentation Outline} % Slide title
	\tableofcontents % Displays the table of contents
\end{frame}

%------------------
%	BODY SECTIONS
%------------------

\section{Text}

\begin{frame}
  \frametitle{Plain text}

  Lorem ipsum dolor sit amet, consectetur adipiscing elit. Curabitur a cursus sapien. Aenean sed mollis mauris. Morbi hendrerit eros id tortor facilisis, at consectetur dui iaculis. Nullam pharetra felis sit amet velit porttitor placerat. In vel lectus ipsum. Aliquam vestibulum a erat et sollicitudin. Quisque volutpat odio finibus nisi pellentesque, a efficitur lorem aliquam. Nunc tellus est, lacinia at mattis sed, vulputate sed neque.  

\end{frame}

\section{Lists}

\begin{frame}
  \frametitle{Bullet list}

  \begin{itemize}
    \item First point
    \item Second point
    \item Third point
  \end{itemize} 

\end{frame}

\begin{frame}{Numbered list}
  \begin{enumerate}
    \item First point
    \item Second point
    \item \begin{enumerate}
      \item Subpoint
    \end{enumerate} 
  \end{enumerate}

\end{frame}



\section{Figures and stuff}

\begin{frame}
  \frametitle{Figure}

  

\end{frame}

\begin{frame}
  \frametitle{Table}

  

\end{frame}
%------------------

\end{document}
